\documentclass[12pt]{article}

\usepackage[utf8]{inputenc}
\usepackage{polski}
\usepackage{verbatim}
\usepackage[polish]{babel}
\usepackage{listings}
\usepackage{textcomp}
\usepackage[T1]{fontenc}

\title{Interpreter prostego języka\\ \large dokumentacja wstępna}
%\subtitle{dokumentacja wstępna}
\author{
        Mariusz Zakrzewski
}
\date{\today}

\begin{document}
\maketitle

\section{Temat i zakres projektu}

Tematem projektu jest interpeter prostego języka programowania, umożliwiającego operacje na dodatkowym typie danych. \\

Język będzie umożliwiał podstawowe operacje na typach danych: liczba całkowita, liczba rzeczywista, napis, zmienna boolowska oraz na dodatkowym 
typie danych reprezentującym poruszający się obiekt, wraz z jego położeniem, prędkością, przyspieszeniem w przestrzeni dwu-wymiarowej. Umożliwiać będzie definiowanie 
własnych funkcji, intrukcji rozgałęziających, pętli, wypisywania na standardowe wyjście.

\section{Wymagania funkcjonalne}
\begin{enumerate}
\item odczytywanie oraz wykonywanie programów zapisanych w plikach tekstowych
\item zgłaszanie błędów kompilacji
\item możliwość definiowania własnych funkcji
\item program dba o poprawną kolejność operatorów na danych
\item język umożliwia pisanie instrukcji rozgałęziających
\item język umożliwia pisanie pętli
\item wyświetlanie komunikatów na standardowe wyjście termianala
\item program dba także o zasięg zmiennych - możliwe jest użycie 2 identycznych nazw w 2 różnych funkcjach - brak konfliktu, możliwe wykorzystanie funkcji rekurencyjnych
\end{enumerate}
\section{Wymagania niefunkcjonalne}
\begin{enumerate}
\item komunikaty o błędach są czytelne
\item czas interpretacji i wykonania jest rozsądnie krótki
\end{enumerate}

\section{Sposób obsługi programu i opis realizacji projektu}
Program zostanie napisany oraz testowany pod systemem Linux w języku C++(standard C++11) jako aplikacja konsolowa. \\
Do zbudowania projektu zostanie użyte narzędzie make lub scons. \\ \\
Struktura projektu będzie obejmowała:
\begin{itemize}
\item moduł lexera(analizatora leksykalnego) odpowiadający za wytworzenie strumienia tokenów z pliku tekstowego
\item moduł parsera(analizatora składniowego) odpowiadający za utworzenie drzewa AST(Abstract Syntax Tree)
\item moduł analizatora semantycznego, odpowiadającego za sprawdzenie powstałego drzewa ast i ewentualne generowanie błędów
\item tablica symboli - klasy pomocnicze reprezentująca zmienne i ich zasięg
\item obsługa błędów - klasy umożliwiające obsługę błędów - wypisanie komunikatu, ewentualną korektę dla procesu wykrycia i wyświetlenia dalszych potencjalnych błędów
\end{itemize}

\section{Przykładowe użycie programu i składnia}

\begin{lstlisting}[columns=fullflexible,basicstyle=\tiny,tabsize=3,]
def main() {
	// komentarz
	/* kom
	entarz
	*/
	
	var myInt: int = 6;
	var myString: string = "hello";
	var myDouble: doube = 3.14;
	var myBoolean: boolean = true;

	//((polozenieX,polozenieY),(predkoscX,predkoscY),(przyspieszenieX,przyspieszenieY))
	var myMoveable: moveable = ((1,2),(3,4),(5,6)); 
	print(myMoveable.posX); // 1
	print(myMoveable.speedX); // 3
	print(myMoveable.accX); // 5 
	var myPosition: position = (2,3);
	print(myPosition.x); // 2
	print(myPosition.y); // 3

	// obiekt singleton globalny dla calego programu, steruje 'czasem' w ruchu wszystkich obiektow
	// zmieniajac wartosc globalTime poruszamy wszystkimi zdefiniowanymi instancjami typu moveable
    // (takze zmieniamy predkosci obiektow o niezerowym przyspieszeniu)
	moveable.globalTime = 0;

	var myMoveable2: moveable = ((0,0),(1,0),(0,0));
	globalTime += 1;
	print(myMoveable2.posX); // 1
	print(myMoveable2.posY); // 0

	var myMoveable3: moveable = ((0,0),(1,0),(0,0));
	var myMoveable4: moveable = ((10,0),(-1,0),(0,0)) ;

	moveable.setOnCollision(myMoveable3, myMoveable4, onCollisionFun);
	
	globalTime += 4;
	globalTime += 1;
	// zostanie zawolana funkcja onCollisionFun(myMoveable3, myMoveable4, (5,0), 6)

	// dodatkowe mozliwosci typu danych moveable, pozwalajace na rozsadne 
    // zarzadzanie tymi obiektami
	myMoveable.id = 12;
	myMoveable.name = "nazwa konkretnego obiektu";

	var myBoolean1: boolean = false;
	if(myBoolean1 && 2 < 4){
		print("true");
	}else{
		print("false");
	}
	
	var iter: int = 0;
    while(iter < 10){
        iter += 1;
        print(iter + " ");
    }

	var a: int = 2;
	var b: int = 3;
	var c: int = 4;
	print(a + b * c); // 14
	print((a + b) * c); // 24

	var myStr: string = "hello";
	var myStr2: string = " world";
	var myStr3: string = myStr + myStr2; // "hello world"
	var myStr4: string = myStr + a; // "hello2"

	/* ==== obsluga bledow === */
	myStr = 3; // type mismatch
	myStr = myStr * 3; // undefined operand '*' for string and int
	if(false {		// ')' expected
		print("abc");
	}
	var myInt5 = 8; // type not specified
	myMoveable.nmae = "abc"; // 'nmae' field not undefined
	// program bez funkcji 'main': function 'main' not defined

	var d: int
	d = sayHello(); // incompatible types: void cannot be converted to int
}

def sayHello() {
	print("hello world");
	// nic nie zwraca
}

def mullBy2(myInt: int): int {
	return myInt * 2
}

def onCollisionFun(mov1: moveable, mov2: moveable, pos: position, time: int) {
	print("zderzenie obiketu: " + mov1.name + ", z obiektem: "
		 + mov2.name + ", w miejscu:" + pos + ", w czasie: " + time)
}
\end{lstlisting}

\section{Gramatyka}
\subsection{Lista tokenów}

def " = , . : ( ) ; \{ \} ( ) while int boolean string double moveable position posX posY speedX speedY accX accY var true false * / + - \% ! \textless \textgreater \textless = \textgreater = += -= *= /= == != \&\& || return null

\subsection{Gramatyka}

\begin{lstlisting}[basicstyle=\tiny,tabsize=3]
function_definition
    : "def" identifier "(" [ formal_arglist ] ")" [":" type ] "{" statement_list "}"

formal_arglist
    : identifier ":" type [ "," formal_arglist ]

statement_list
    : statement
    | (statement statement_list)

statement
    : compound_statement
    | if_statement
    | while_statement
    | expression ";"
    | "return"
    | "return" expression;
    | "var" identifier ":" type "=" expression ";"
    | "var" identifier ":" type ";"

compound_statement:
    : ("{" "}")
    | ("{" statement_list "}")

if_statement
    : "if" "(" expression ")" statement [ "else" statement]

while_statement
    : "while" "(" expression ")" statement

expression
    : numeric_constant
    | string_constant
    | "(" "(" expression "," expression ")" "," "(" expression "," expression ")" "," "(" expression "," expression ")" ")"
    | "(" expression "," expression ")"
    | "null"
    | identifier
    | identifier "(" actual_arglist ")"
    | identifier "." identifier
    | expression binary_operator expression
    | unary_operator expression
    | "(" expression ")"
    | identifier assignment_operator expression
    | identifier "." identifier assignment_operator expression

actual_arglist
    : expression [ "," actual_arglist]

binary_operator
    : "+" | "-" | "*" | "/" | "%" | "<" | "<=" | "==" | "=>" | ">"

unary_operator
    : "!" | "-" | "&"

assignment_operator
    : "+=" | "*=" | "-=" | "/=" | "="

numeric_constant
    : (1-9)[0-9]* [ "." (0-9)*]

string_constant
    : (a-Z | "_")*
\end{lstlisting}

\end{document}
  